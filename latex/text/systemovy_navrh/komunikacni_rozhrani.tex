\section{Komunikační rozhraní}
\label{sec:komunikacni-rozhrani}
    Před návrhem jednotlivých částí zařízení je zapotřebí definovat komunikační rozhraní mezi řídicím modulem a periferiemi, protože právě od jeho specifikace se následně odvíjí tvorba zbytku zařízení. 

    Úkolem rozhraní je obousměrně komunikovat s~periferiemi, tedy např. stahovat data z~připojených sensorů a zároveň za pomoci příkazů periferie řídit. Kromě datové komunikace musí rozhraní periferie také napájet, a to i v~případě energeticky náročnějších obvodů jako např. osvětlení. 

    Pro připojení periferií byl zvolen konektor typu D-sub 9, který je cenově dostupný, disponuje dostatečným množstvím pinů a umožňuje montáž jak do panelu, tak i jako zakončení kabelu. Přiřazení a funkce jednotlivých vodičů jsou vyobrazeny na obr.~\ref{fig:komunikacni-rozhr-dsub-pinout}. Konektor disponuje dvěma úrovněmi napájení, \qty{5}{V} slouží k~napájení mikrokontrolérů modulů periferií a k~nim připojených senzorů nebo jiné nenáročné elektroniky. Je pravděpodobné, že s~připojením více periferií za sebe dojde k~úbytku napětí v~důsledku ztrát na vedení. Aby bylo toto částečně kompenzováno, napětí vystupující z~měniče řídicí jednotky je přibližně o~\qty{0.2}{V} vyšší. Druhou napájecí linkou je výstup přímo z~externího spínaného zdroje, tedy s~napětím \qty{24}{V}, ten slouží pro výkonově náročnější periferie, které si již napětí dále upraví podle potřeby a nebudou neúměrně zatěžovat první zmíněnou napájecí linku.

    \begin{figure}[h!]
        \centering
        % trim=left bottom right top
        \begin{tikzpicture}

            % Draw the connector outline
            \draw[rounded corners=5pt] (0,0) -- (2,0) -- (2.6,1.5) -- (-0.6,1.5) -- cycle;
            
            % Draw the pins
            \newcounter{pinnum}
            \setcounter{pinnum}{9}
            \def\xoff{-0.25}
            \foreach \x in {0.5,1,...,2}
                \foreach \y in {0.6}
                {
                    \draw (\xoff+\x,\y) circle (0.1);
                    \node[anchor=north,font=\fontsize{9}{0}\selectfont] at (\xoff+\x,\y-0.03) {\thepinnum};
                    \setcounter{pinnum}{\value{pinnum}-1}
                }

            \def\xoff{-0.5}
            \foreach \x in {0.5,1,...,2.5}
                \foreach \y in {1.0}
                {
                    \draw (\xoff+\x,\y) circle (0.1);
                    \node[anchor=south,font=\fontsize{9}{0}\selectfont] at (\xoff+\x,\y+0.03) {\thepinnum};
                    \setcounter{pinnum}{\value{pinnum}-1}
                }
            
            % Pin numbers
            \node[anchor=west] (pin1) at (-4,1.4)                            {1,~5,~6,~9: GND};
            \node[below=0.6 of pin1.west,anchor=west] (pin2) {~~~~~~2,~7: \qty{24}{V}};
            \node[below=0.6 of pin2.west,anchor=west] (pin3) {~~~~~~~~~8: \qty{5.2}{V}};
            \node[anchor=west] at (2,1.1) (pin4) {~~~~~~~~~3: \acs{can} High};
            \node[below=0.6 of pin4.west,anchor=west] (pin5) {~~~~~~~~~4: \acs{can} Low};
            
        \end{tikzpicture}
        \caption{Přiřazení pinů konektorů D-sub 9 pro připojení periferií.}
        \label{fig:komunikacni-rozhr-dsub-pinout}
    \end{figure}

    
    \subsection{Výběr datové sběrnice}
        Existuje celá řada datových sběrnic, které jsou v~elektrotechnice hojně využívány. Každá z~nich má své výhody a nevýhody, stejně tak jako jisté limitace použití. V~tab.~\ref{tab:porovnani-sbernic} se nachází výčet různých sběrnic, které byly při výběru uvažovány. 

        \begin{table}[h]
            \centering
            \caption{Datové sběrnice, porovnání~\cite{prodigy-spi-i2c}.}
            \label{tab:porovnani-sbernic}
            \begin{tabularx}{\textwidth}{|p{1.3cm}|X|X|X|}
            \hline
            \textbf{Typ} & \textbf{Výhody} & \textbf{Nevýhody} & \textbf{Limitace} \\
            \hline\hline
            \acs{spi} & 
            \begin{tabular}[t]{@{}p{4cm}@{}}
            Více zařízení na sběrnici \\
            Vysoká rychlost přenosu dat \\
            Jednoduchý protokol \\
            \end{tabular} &
            \begin{tabular}[t]{@{}p{4cm}@{}}
            Nutný CS pin pro každé zařízení \\
            \end{tabular} &
            \begin{tabular}[t]{@{}p{4cm}@{}}
            Určeno na krátkou vzdálenost \\
            \end{tabular} \\
            \hline
            I\(^{2}\)C &
            \begin{tabular}[t]{@{}p{4cm}@{}}
            Pouze 2 piny \\
            Více zařízení -- 128 adres \\
            \end{tabular} &
            \begin{tabular}[t]{@{}p{4cm}@{}}
            Riziko kolize adres \\
            Nižší rychlost přenosu dat proti \acs{spi} \\
            \end{tabular} &
            \begin{tabular}[t]{@{}p{4cm}@{}}
            Určeno na krátkou vzdálenost \\
            \end{tabular} \\
            \hline
            \acs{can} &
            \begin{tabular}[t]{@{}p{4cm}@{}}
            Vysoká spolehlivost \\
            Dlouhé propojení \\
            \end{tabular} &
            \begin{tabular}[t]{@{}p{4cm}@{}}
            Vyšší náklady na implementaci \\
            Nižší rychlost přenosu dat \\
            \end{tabular} &
            \begin{tabular}[t]{@{}p{4cm}@{}}
            Nepodporovano běžnými \acs{mcu} -- nutný externí řadič \\
            \end{tabular} \\
            \hline
            UART &
            \begin{tabular}[t]{@{}p{4cm}@{}}
            Jednoduchá implementace \\
            Možnost asynchronní komunikace \\
            \end{tabular} &
            \begin{tabular}[t]{@{}p{4cm}@{}}
            Nižší rychlost přenosu dat proti \acs{spi} \\
            Pouze 2 zařízení \\
            \end{tabular} &
            \begin{tabular}[t]{@{}p{4cm}@{}}
            Pouze 2 zařízení \\
            Určeno na krátkou vzdálenost \\
            \end{tabular} \\
            \hline
            \end{tabularx}
            
        \end{table}


        Hlavní šasi zařízení nabízí dva konektory. Během provozu je ale žádoucí připojit větší, předem nedefinovaný počet periferií. Proto je potřeba, aby zvolená sběrnice umožnila připojení více zařízení současně. Obecný problém všech sběrnic je omezení jejich maximální délky. S~rostoucí délkou se sběrnice snáze zaruší, navíc z~důvodu parazitních vlastností vedení dochází k~zaoblení ostrých hran signálu, dlouhé vedení se chová jako filtr typu dolní propust. V~důsledku toho se snižuje maximální rychlost sběrnice.
        
        Sběrnice \acs{spi} nebo \acs{i2c} je obecně doporučeno používat pouze v~rámci \acs{dps}, tedy na krátké vzdálenosti. Při snížení rychlosti je možné je používat i na větší vzdálenost, ovšem modulární scénář vytvářeného systému teoreticky nestanovuje žádný délkový limit a bylo by velmi obtížné spolehlivě určit, kolik periferií uživatel může za sebe zapojit při zachování spolehlivé komunikace.

        UART je výhodný svou jednoduchou implementací a umožňuje obousměrnou asynchronní komunikaci. Nevýhodou je, že funguje pouze pro dvě zařízení. Jednou z~možností, jak tuto limitaci obejít, by bylo zavedení řetězového způsobu komunikace, kdy by každé zařízení komunikovalo se dvěmi sousedními a informace by se postupně předávala dále až k~cílovému zařízení. Tento systém je relativně jednoduchý, ale například v~případě poruchy jednoho zařízení se odpojí všechna následující zařízení, což může mít neočekávané následky.

        Sběrnice \acs{can} je určena pro provoz v~průmyslovém prostředí (zejména je používána v~automobilovém průmyslu) a díky své robustnější konstrukci ji lze bez problému použít i na delší vzdálenosti a pro více zařízení. Při komunikaci je používán diferenční pár vodičů, takže i odolnost proti rušení je výrazně lepší. Nevýhodou je ale její o~něco složitější a dražší implementace. Většina běžných mikrokontrolérů nemá pro \acs{can} vestavěnou periferii a je tak potřeba buďto zvolit dražší mikrokontrolér nebo připojit externí ovladač (řízený např. přes \acs{spi}). Dále je nutné přidat i řadič, který převede signál na diferenční a zároveň umožní zvýšit provozní napětí na 12  nebo \qty{24}{V}, čímž dojde k~ještě lepšímu potlačení šumu.

    \subsection{Sběrnice \acs{can}}
        Po důkladné rešerši a zvážení zmíněných kladů a záporů byla  zvolena sběrnice \acs{can}. ESP32 jakožto již zvolený mikrokontrolér řídicí jednotky obsahuje vestavený \acs{can} kontrolér a pro moduly periferií byl na základě tohoto rozhodnutí zvolen také vhodný mikrokontrolér. Co se týče nutnosti přidání řadiče, jedná se sice o~další součástku, která na první pohled navyšuje cenu zařízení, kromě převodu signálu na diferenční ale zajišťuje také ochranu konektorů proti mnoha nežádoucím jevům jako je zkrat, ESD výboj nebo přepětí. Tímto se ve výsledku celé zapojení zlevní a zjednoduší.


   
