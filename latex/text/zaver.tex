\chapter*{Závěr}
\phantomsection
\addcontentsline{toc}{chapter}{Závěr}

V~rámci bakalářské práce bylo navrženo a sestaveno zařízení, určené k ovládání běžného domácího akvária. Jedná se o modulární systém sestávající z řídicí jednotky a několika propojených modulů, které vzájemně komunikují po sběrnici CAN. K systému byla také vytvořena jednoduchá webová stránka, přes kterou je možné systém vzdáleně konfigurovat a monitorovat. 

Teoretická část práce se věnuje problematice provozu akvária. Jsou zde rozebrány důležité veličiny, které je potřeba monitorovat a ovládat pro spolehlivé přežití akvarijního ekosystému. Dále se text věnuje průzkumu trhu a používané akvarijní technice. Postupováno je od nezbytného minima pro založení malého akvária, až po systémy zajišťující komplexní automatizaci velkých instalací, složených z~více nádrží.

Na základě provedené rešerše jsou stanoveny požadavky na podobu a funkci vlastního zařízení. Jeho cílem není konkurovat svými možnostmi drahým pokročilým systémům, ale spíše dosáhnout jistého kompromisu mezi cenou a stále dostatečně širokou funkcionalitou k~automatizaci menšího domácího akvária. Samotný návrh systému začíná popisem zvolené architektury, poté jsou podrobněji popsány jednotlivé dílčí části. Byly navrženy tři desky plošných spojů. Řídící jednotka obsahuje spínaný zdroj použitý k napájení celého systému a mikrokontrolér ESP32 zajišťující kromě samotného řízení také komunikaci skrze síť WiFi. Moduly periferií jsou navrženy jako univerzální platforma, ke které lze připojit různé obvody sensorů nebo akčních členů. Pro plynulé ovládání LED osvětlení byla navržena vlastní deska, kterou lze do této platformy vložit. Ostatní periferie již nebyly tak komplexní a pro připojení sensorů k univerzální platformě postačila prototypová deska. 

Zvolený způsob komunikace periferií a ovládání přes internet učinil systém velmi komplexní a proto při tvorbě softwaru vznikla řada překážek, kterým bylo potřeba čelit. Klíčové softwarové bloky byly zvládnuty úspěšně. Jednotlivé části systému mezi sebou vzájemně komunikují a možná je i konfigurace systému přes webové rozhraní, ta je navíc zachována v paměti flash i po restartu zařízení. Stejně tak je zařízení schopné odesílat naměřená data, která si uživatel může na webu zobrazit. Samotný algoritmus řízení akvária ale zatím není dostačující k úplnému a spolehlivému provozu a řadu funkcí bude potřeba dokončit. 

Jednotlivé dílčí části zařízení byly testovány postupně, z důvodu nalezených chyb bude dlouhodobý test v provozu akvária teprve následovat. 

Volba modulární architektury sice návrh zařízení značně zkomplikovala a díky tomuto rozhodnutí nebylo dosaženo všech vytyčených cílů, na druhou stranu je ale zařízení velmi flexibilní a do budoucna má potenciál sloužit nejen jako systém řízení akvária. S drobným rozšířením může snadno obsloužit vícero oblastní domácí automatizace a stát se tak součástí moderního fenoménu tzv. Smart Home.