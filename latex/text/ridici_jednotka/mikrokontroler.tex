\section{Mikrokontrolér}

    Při výběru vhodného mikrokontroléru bylo potřeba zohlednit výše zmíněné požadavky, tedy zejména Wi-Fi konektivitu a dostatečný výkon k~její obsluze, periferii \acs{can} a dostatek \acs{gpio} pinů pro připojení zbylých modulů v~hlavním šasi (viz obr.~\ref{fig:blokove-schema}). Na trhu existuje vícero výrobců nabízejících mikrokontroléry s~vhodnými parametry, z~důvodu jednoduchosti použití a nízké ceny byl nakonec zvolen model ESP32 od firmy Espressif, konkrétně modul WROOM-32E~\cite{esp32-wroom-32e-datasheet} s~čipem ESP32-D0WDR2-V3~\cite{esp32-datasheet}. Tento modul je často využíván v~různých hobby projektech, ale také v~komerčních aplikacích zejména v~oblasti chytré domácnosti. Z~tohoto důvodu k~němu existuje velká škála softwarových knihoven a v~rámci komunity uživatelů je také sdíleno mnoho projektů, kterými je možné se inspirovat.