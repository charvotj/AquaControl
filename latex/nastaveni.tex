% V tomto souboru se nastavují téměř veškeré informace, proměnné mezi studenty:
% jméno, název práce, pohlaví atd.
% Tento soubor je SDÍLENÝ mezi textem práce a prezentací k obhajobě -- netřeba něco nastavovat na dvou místech.

\usepackage[
%%% Z následujících voleb jazyka lze použít pouze jednu
  czech-english,		% originální jazyk je čeština, překlad je anglicky (výchozí)
  %english-czech,	% originální jazyk je angličtina, překlad je česky
  %slovak-english,	% originální jazyk je slovenština, překlad je anglicky
  %english-slovak,	% originální jazyk je angličtina, překlad je slovensky
%
%%% Z následujících voleb typu práce lze použít pouze jednu
  % semestral,		  % semestrální práce (výchozí)
  bachelor,			%	bakalářská práce
  %master,			  % diplomová práce
  %treatise,			% pojednání o disertační práci
  %doctoral,			% disertační práce
%
%%% Z následujících voleb zarovnání objektů lze použít pouze jednu
%  left,				  % rovnice a popisky plovoucích objektů budou zarovnány vlevo
	center,			    % rovnice a popisky plovoucích objektů budou zarovnány na střed (vychozi)
%
]{thesis}   % Balíček pro sazbu studentských prací


%%% Jméno a příjmení autora ve tvaru
%  [tituly před jménem]{Křestní}{Příjmení}[tituly za jménem]
% Pokud osoba nemá titul před/za jménem, smažte celý řetězec '[...]'
\author{Jakub}{Charvot}

%%% Identifikační číslo autora (VUT ID)
\butid{240844}

%%% Pohlaví autora/autorky
% (nepoužije se ve variantě english-czech ani english-slovak)
% Číselná hodnota: 1...žena, 0...muž
\gender{0}

%%% Jméno a příjmení vedoucího/školitele včetně titulů
%  [tituly před jménem]{Křestní}{Příjmení}[tituly za jménem]
% Pokud osoba nemá titul před/za jménem, smažte celý řetězec '[...]'
\advisor[Ing.]{Pavel}{Tomíček}

%%% Jméno a příjmení oponenta včetně titulů
%  [tituly před jménem]{Křestní}{Příjmení}[tituly za jménem]
% Pokud osoba nemá titul před/za jménem, smažte celý řetězec '[...]'
% Nastavení oponenta se uplatní pouze v prezentaci k obhajobě;
% v případě, že nechcete, aby se na titulním snímku prezentace zobrazoval oponent, pouze příkaz zakomentujte;
% u obhajoby semestrální práce se oponent nezobrazuje (jelikož neexistuje)
% U dizertační práce jsou typicky dva až tři oponenti. Pokud je chcete mít na titulním slajdu, prosím ručně odkomentujte a upravte jejich jména v definici "VUT title page" v souboru thesis.sty.
% TODO:
\opponent[Ing.]{Vladimír}{Levek}[Ph.D.]

%%% Název práce
%  Parametr ve složených závorkách {} je název v originálním jazyce,
%  parametr v hranatých závorkách [] je překlad (podle toho jaký je originální jazyk).
%  V případě, že název Vaší práce je dlouhý a nevleze se celý do zápatí prezentace, použijte příkaz
%  \def\insertshorttitle{Zkác.\ náz.\ práce}
%  kde jako parametr vyplníte zkrácený název. Pokud nechcete zkracovat název, budete muset předefinovat,
%  jak se vytváří patička slidu. Viz odkaz: https://bit.ly/3EJTp5A
\title[Autonomous system for control of aquarium]{Autonomní systém pro řízení akvária}

%%% Označení oboru studia
%  Parametr ve složených závorkách {} je název oboru v originálním jazyce,
%  parametr v hranatých závorkách [] je překlad
\specialization[Microelectronics and Technology]{Mikroelektronika a technologie}

%%% Označení ústavu
%  Parametr ve složených závorkách {} je název ústavu v originálním jazyce,
%  parametr v hranatých závorkách [] je překlad
%\department[Department of Control and Instrumentation]{Ústav automatizace a měřicí techniky}
%\department[Department of Biomedical Engineering]{Ústav biomedicínského inženýrství}
%\department[Department of Electrical Power Engineering]{Ústav elektroenergetiky}
%\department[Department of Electrical and Electronic Technology]{Ústav elektrotechnologie}
%\department[Department of Physics]{Ústav fyziky}
%\department[Department of Foreign Languages]{Ústav jazyků}
%\department[Department of Mathematics]{Ústav matematiky}
\department[Department of Microelectronics]{Ústav mikroelektroniky}
%\department[Department of Radio Electronics]{Ústav radioelektroniky}
%\department[Department of Theoretical and Experimental Electrical Engineering]{Ústav teoretické a experimentální elektrotechniky}
% \department[Department of Telecommunications]{Ústav telekomunikací}
%\department[Department of Power Electrical and Electronic Engineering]{Ústav výkonové elektrotechniky a elektroniky}

%%% Označení fakulty
%  Parametr ve složených závorkách {} je název fakulty v originálním jazyce,
%  parametr v hranatých závorkách [] je překlad
%\faculty[Faculty of Architecture]{Fakulta architektury}
\faculty[Faculty of Electrical Engineering and~Communication]{Fakulta elektrotechniky a~komunikačních technologií}
%\faculty[Faculty of Chemistry]{Fakulta chemická}
%\faculty[Faculty of Information Technology]{Fakulta informačních technologií}
%\faculty[Faculty of Business and Management]{Fakulta podnikatelská}
%\faculty[Faculty of Civil Engineering]{Fakulta stavební}
%\faculty[Faculty of Mechanical Engineering]{Fakulta strojního inženýrství}
%\faculty[Faculty of Fine Arts]{Fakulta výtvarných umění}
%
%Nastavení logotypu (v hranatych zavorkach zkracene logo, ve slozenych plne):
\facultylogo[logo/FEKT_zkratka_barevne_PANTONE_CZ]{logo/UMEL/CZ/PDF/UMEL_color_PANTONE_CZ}

%%% Rok odevzdání práce
\graduateyear{2024}
%%% Akademický rok odevzdání práce
\academicyear{2023/24}

%%% Datum obhajoby (uplatní se pouze v prezentaci k obhajobě)
\date{11.\,6.\,2024} 

%%% Místo obhajoby
% Na titulních stránkách bude automaticky vysázeno VELKÝMI písmeny (pokud tyto stránky sází šablona)
\city{Brno}

%%% Abstrakt
% TODO:
\abstract[%
This thesis delves into the topic of~aquarium automation, aiming to~design custom system for this purpose. The~essential technology requirements for~efficient aquarium operation are summarized at the beggining followed by the~market survey with depiction of~existing commercial solutions in~the~field of~automation. Regarding the practical part, the thesis explain the design process of the device and takes a~deeper look at its crutial steps. System architecture is disccused as well as the creation of electrical schematics, and the design of custom printed circuit boards. The final part of the text is dedicated to the software. The outcome of the thesis is a modular system consisting of a control unit and several connected peripherals managing specific sensors and actuators. The entire system can be configured remotely via a web application.
]{%
Tato práce se zaměřuje na problematiku automatizace akvárií a jejím cílem je navrhnout vlastní systém sloužící tomuto účelu. V~práci jsou shrnuty technické požadavky na provoz akvária a je proveden průzkum trhu se zaměřením na existující komerční řešení v oblasti automatizace. Praktická část práce detailně popisuje návrh zařízení a jeho jednotlivé fáze. Je popsána architektura na úrovni funkčních bloků, tvorba elektrických schémat i návrh desek plošných spojů. Poslední část je pak věnována softwaru. Výstupem práce je modulární systém sestávající z řídící jednotky a několika připojených periferií obsluhujících konrétní sensory a akční členy. Celý systém je možné konfigurovat vzdáleně pomocí webové aplikace. 
}

%%% Klíčová slova
% TODO:
\keywrds[%
aquaristics, automation, ESP32, CAN bus, device design, printed circuit boards
]{%
akvaristika, automatizace, ESP32, sběrnice CAN, návrh zařízení, desky plošných spojů
}

%%% Poděkování
% TODO:
\acknowledgement{%
Rád bych poděkoval vedoucímu své bakalářské práce
panu Ing.~Pavlu Tomíčkovi\ za všudypřítomný optimismus a ochotu konzultovat mé problémy kdykoliv bylo potřeba. Dále děkuji svému kamarádovi Radku Jančičkovi za osvětu v~oblasti akvaristiky. V neposlední řadě také děkuji své přítelkyni za trpělivé snášení hluku přístrojů a neustále se hromadících součástek. 

}%