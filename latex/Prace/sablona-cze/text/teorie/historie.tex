\section{Historie}
\subsection{Počátky}
Akvaristika v~různých podobách provází lidstvo téměř od prvopočátku. Nejprve se jednalo spíše o~chov ryb užitkových, tedy rybářství, ovšem už ve starověké Mezopotámii docházelo také k~chovu ryb okrasných. Počátky akvaristiky byly prováděny spíše metodou pokusů a omylů, protože lidem nebyla známa velká část přírodních zákonitostí -- životní potřeby chovaných ryb, způsob jejich rozmnožování a v~neposlední řadě také procesy, odehrávající se v~přírodním ekosystému, zajišťující jeho rovnováhu. Základem udržení chovaných ryb naživu byla zejména častá výměna vody, ani tak ale dlouho nebylo možné udržet ryby při životě dlouhodobě. 

V~období středověku se poprvé objevuje také dovoz exotických okrasných rybek z~cizích zemí, pro naprostý nedostatek znalostí ale často brzo hynou, např. jen proto, že chovatele nenapadne je nakrmit~\cite{vitek_akvaristika}.

\subsection{Věda a technika}
Na konci 18. století dochází k~rozvoji vědy a několika objevům, které historii akvaristiky zásadně ovlivnily. Poprvé byl izolován kyslík, byl objasněn princip dýchání živočichů a následně také fotosyntéza. Akvaristika, v~tehdejší době umělý chov ryb za účelem pozorování a výzkumu, byla provozována zejména na vědecké půdě a byl zde zájem o~zdokonalení používaných technik a postupů. V~roce 1837 S. H. Ward prakticky prokázal, že osvětlené akvárium obsahující jak rybky, tak i rostliny, vydrží velmi dlouho bez nutnosti výměny vody~\cite{vitek_akvaristika}. Pricip výměny plynů byl významným milníkem ve snaze dosáhnout v~akváriu rovnováhy podobné přírodnímu prostředí. 

Při stále nových poznatcích o~životních potřebách ryb a o~akvarijní rovnováze bylo nutné přijít s~různými technickými řešeními. Akvária 19. a 20. století už byla vytápěná a uměle okysličovaná. Původní mechanická řešení a lihové kahany byly postupně nahrazovány elektrickými přístroji. V~pozdějších letech pak přibylo i umělé osvětlení a systémy filtrace vody. 