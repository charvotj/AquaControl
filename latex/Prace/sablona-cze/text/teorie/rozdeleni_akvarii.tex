\section{Rozdělení akvárií}
Akvária je možné rozdělit na základě mnoha různých parametrů jako je např. velikost, materiál a tvar anebo jejich funkce. Pro účely této práce jsou však relevantní zejména rozdělení, která jsou zásadní pro rozsah použité akvaristické techniky. 

V jednoduchosti tedy můžeme akvária rozdělit podle biotopu~\cite{haskova_bakalarska2011}:
\begin{itemize}
    \item Sladkovodní
    \item Brakická -- salinita přibližně 5 až \qty{15}{\permille}
    \item Mořská -- salinita přibližně 30 až \qty{40}{\permille}
\end{itemize}

Asi není potřeba vysvětlovat, že pro akvária mořská a brakická nestačí použít běžnou kohoutkovou vodu, ale je potřeba ji před použitím upravit. Pokud chceme systém automatizovat, je potřeba přidat zařízení, které bude salinitu průběžně monitorovat a upravovat. Komplexní profesionální systémy (např. GHL, Neptune Apex, ...) tyto možnosti nabízejí, ale pořizovací cena je relativně vysoká. Můžeme tedy říci, že po technické stránce je provoz sladkovodních akvárií jednodušší než provoz akvárií mořských. 

Další dělení akvárií je možné z hlediska jejich obsazení:
\begin{itemize}
    \item Čistě rostlinná akvária
    \item S běžnými druhy ryb
    \item Se speciálními druhy -- zvýšené nároky na parametry vody
\end{itemize}
Rozsah použité akvaristické techniky a zejména požadavek na její přesnost je závislý na volbě umístěných druhů rostlin a živočichů. Každý druh má své optimální životní podmínky a zatímco některým živočichům se bude dařit ve vodě o teplotě v rozsahu klidně i \qty{15}{\degreeCelsius}, jiné vyžadují téměř konstantní teplotu v rozsahu třeba jen \qty{2}{\degreeCelsius}, to zásadně ovlivní požadavky na přesnost měření teploty i způsob její regulace. Stejně tak je tomu i s dalšími parametry.

Zařízení vytvořené v rámci této práce bude určeno pro použití v menším sladkovodním akváriu osazeném běžnými druhy rostlin a živočichů bez speciálních životních potřeb -- tedy scénář běžného domácího akvaristy s omezeným rozpočtem. Není ale vyloučeno jeho budoucí rozšíření i pro náročnejší aplikace.