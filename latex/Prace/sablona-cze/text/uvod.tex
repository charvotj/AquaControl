\chapter*{Úvod}
\phantomsection
\addcontentsline{toc}{chapter}{Úvod}


V dnešní době, kdy jsou na vzestupu fenomény jako chytrá domácnost, \acs{iot} (\acl{iot}) nebo Průmysl 4.0, se na trhu objevuje stále více výrobků, které se snaží automatizovat a zjednodušit různé oblasti našeho života. Tento trend se dnes dotýká nejedné volnočasové aktivity a to včetně akvaristiky. Tu lze samozřejmě provozovat na různé úrovni, ale i majitelé malých domácích akvárií potřebují k provozu svého koníčku relativně velké množství elektroniky. 
Běžnou praxí je, že každé z použitých zařízení je ovládáno buďto zcela ručně nebo, pokud disponuje možností vzdáleného přístupu a automatizace, má svou samostanou aplikaci a uživatel tak provoz akvária musí ovládat z několika různých míst, což může být značně nepohodlné a nepřehledné.

Na trhu samozřejmě existují také velmi sofistikované a komplexní systémy, ty ovšem svou cenou vysoce přesahují rozpočet běžného \uv{domácího} akvaristy. Tato práce se věnuje návrhu a tvorbě zařízení, které má za cíl nabídnout pohodlnou kontrolu a ovládání všech potřebných součástí domácího akvária, a to při zachování jednoduchosti a nízké pořizovací ceny.
