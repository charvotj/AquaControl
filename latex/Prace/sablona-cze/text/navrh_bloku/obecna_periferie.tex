\section{Obecný modul periferie}
    % \textit{    TODO: vybrán MCU PIC16F15325 -- jednoduché zapojení + znám použití ze školy, cenově vychází nejlépe, když zohledníme požadavky na periferie (2x UART, PWM). } 

    % \textit{
    %     Moje představa je navrhnout jednoduchou univerzální desku, která bude mít PICku, LDO a napojené vstupní a výstupní UARTy + ochrany na výstupech, k ní budou pin headery a možnost vložit shield/dauther board, který bude mít případnou další elektroniku k obsluze senzoru atd. Když nebudu stíhat, což určitě nebudu :), tak může být třeba na prototypové desce, což bude asi rozumnější než objednávat 10 různých DPS a pak v každé řešit chyby.} 

    Díky zvolené koncepci systému je možné za periferii považovat jakékoliv zařízení schopné obousměrně komunikovat po navržené sběrnici. Není vyloučeno, aby byla každá periferie navržena zcela odlišně na základě svých vlastních požadavků na výkon, počet pinů nebo dostupná rozhraní daného MCU. Hlavní výhodou této koncepce je to, že periferie mohou být vyvíjeny postupně a přidávány do již funkčního a odladěného systému bez nutnosti modifikovat stávájící hardware. V~případě chyby v~návrhu periferie je také oprava méně náročná, než by tomu bylo v~případě zabudování veškeré funkcionality přímo do řídící jednotky. 

    Nicméně pokud by byl pro každou periferii zvolen zcela jiný mikrokontroler a vytvořen vlastní návrh DPS, vývoj více periferií by byl zbytečně drahý a časově náročný. Proto byl zvolen koncept \uv{obecného modulu periferie}, tedy jedné DPS s~konkrétním mikrokontrolerem zajišťující připojení k~oběma stranám komunikačního rozhraní, napájení periferie a rozhraní pro programování. Kromě toho budou na DPS dvě dutinkové lišty, do kterých bude možné vsadit druhou DPS (popř. během vývoje pouze prototypovací desku) ve funkci dceřinné desky (ang. daughterboard). Vložená deska pak bude obsahovat obvody nutné přímo pro danou konkrétní periferii, např. pro teploměr to bude elektronika umožňující připojení teplotního čidla k~mikrokontroleru. 

    V~aktuální fázi tento práce byl pouze zvolen vyhovující mikrokontroler, návrh konkrétního schématu a rozložení DPS bude předmětem práce budoucí. 
    
    \subsection{MCU}

        Kritéria pro výběr mikrokontroleru byla následující:
        \begin{itemize}
            \item Musí nutně splňovat:
            \begin{itemize}
                \item 2x UART periferie -- pro komunikaci po sběrnici 
                \item PWM výstup -- řízení LED, popř. jiné
                \item Nízká cena 
            \end{itemize}
            \item Je výhodou:
            \begin{itemize}
                \item Dobrá dokumentace, komunita uživatelů
                \item Zkušenost autora s~danou platformou
                \item Další periferie (\acs{i2c}, SPI, ...)
            \end{itemize}
        \end{itemize}

        Na základě těchto kritérií byl vybrán mikrokontroler \textbf{PIC16F15325} od firmy Microchip, ten splňuje všechna kritéria a disponuje také množstvím dalších periferií, které by mohly být v~budoucnu užitečné~\cite{PIC16F15325}. 