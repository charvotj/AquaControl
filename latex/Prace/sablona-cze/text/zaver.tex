\chapter*{Závěr}
\phantomsection
\addcontentsline{toc}{chapter}{Závěr}

V~rámci semestrání práce byla rozebrána akvaristická technika používaná napříč všemi úrovněmi zkušeností uživatelů, od absolutně nezbytného minima pro založení malého domácího akvária až po systémy zajišťující co nejlépe automatizovaný provoz velkých instalací skládajících se z~více nádrží. Tato rešerše sloužila pro lepší orientaci čtenáře, ale i samotného autora, v~problematice provozu a automatizace akvárií. 

Na základě provedené rešerše tato práce stanovuje a upřesňuje požadavky na podobu a funkce vlastního zařízení. Jeho cílem není konkurovat svými možnostmi drahým pokročilým systémům, ale spíše dosáhnout jistého kompromisu mezi cenou a stále dostatečně širokou funkcionalitou k~automatizaci menšího domácího akvária.

V~praktické části práce byl proveden systémový návrh na úrovni funkčních bloků a popsány vztahy mezi nimi. Jelikož cílem autora bylo dosažení co největší míry modularity a rozšiřitelnosti systému, byl kladen velký důraz na promyšlení komunikačního rozhraní mezi řídící jednotkou a periferiemi. V~práci je také popsán koncept \uv{obecné periferie}, kterého se autor plánuje dále držet a který po dokončení a doladění chyb velmi výrazně urychlí vývoj jakékoliv periferie. Dále bylo vytvořeno elektrické schéma pro řídící jednotku včetně napájecího obvodu a ošetření konektorů proti zkratu a přepětí. 

V~rámci navazující bakalářské práce bude na základě schématu vytvořeno rozložení DPS a dokončen návrh modulu \uv{obecné periferie}, opět včetně schématu a rozložení DPS. Volbou modulární architektury se nicméně náročnost současného návrhu o~něco zvýšila a nezůstal tak prostor pro výběr konkrétních senzorů a akčních členů, ten bude tedy také předmětem navazující práce. Finálním krokem pak bude sestavení a naprogramování celého systému a jeho následné otestování v~praxi.
