\section{Požadavky}
\label{sec:pozadavky}
    Cílem je vytvořit zařízení, které umožní co nejvíce automatizovat provoz akvária. Hlavním aspektem by měla být jednoduchost použití pro koncového uživatele, vše by mělo být nanejvýš intuitivní a přehledné. Zařízení musí mít možnost připojení k~internetu prostřednictvím sítě Wi-Fi, uživatel tak bude moci zařízení konfigurovat a sledovat z~libovolného místa za pomoci webové stránky popř. mobilní aplikace.

    Požadavky jednotlivých akvaristů se mohou lišit a zároveň se v~čase měnit. Vytvoření dokonalého a všestaranného zařízení, které vyhoví všem účelům použití není v~časových ani finančních možnostech bakalářské práce, proto byl stanoven požadavek, aby bylo zařízení co nejvíce modulární a rozšiřitelné. Musí být zvolena taková architektura, aby bylo možné v~budoucnu přidat další funkce a periferie bez nutnosti modifikovat stávající hardware.

    Výstupem bakalářské práce by mělo být zařízení schopné monitorovat některé akvaristické veličiny a na základě jejich hodnoty informovat uživatele a ovládat akvárium. Zařízení bude přímo řídit LED páskové osvětlení na 12 nebo \qty{24}{V} a spínat popř. vypínat již existující akvaristické přístroje pracující se síťovým napětím \qty{230}{V}.  

    Jelikož modulární architektura bude nepochybně vyžadovat použití více než jednoho mikrokontroleru a tedy také více různých firmwarů, je potřeba zajistit jejich vzájemnou kompatibilitu a stabilitu celého systému. Veškerý firmware tak musí být verzovaný a po připojení nové periferie musí řídící jednotka rozpoznat, o~jakou periferii se jedná. V~případě připojení nekompatibilní periferie (např. z~důvodu zastaralého firmwaru řídící jednotky) musí být uživatel upozorněn a nesmí být nijak narušena funkce zbytku systému. Aby bylo možné těmto situacím předejít, musí mít řídící jednotka možnost vzdálené aktualizace firmwaru.