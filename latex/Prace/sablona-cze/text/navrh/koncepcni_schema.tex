\section{Blokové schéma}



\begin{figure}[h!]
    \centering
    \begin{tikzpicture}[
        node distance=2cm,
        blok/.style={draw, rectangle, rounded corners=8pt, minimum height=1.5cm, minimum width=3cm},
        rect/.style={draw, dashed, blue, inner sep=15pt, fit=#1},
        label/.style={blue}
    ]
    
        \node (napajeni) [blok] {Napájecí zdroj 24V};
        \node (ridici) [blok, below of = napajeni, align=center, fill=yellow] {Řídící jednotka \\ (ESP32)};
        \node (rele) [blok, right=0.5 of ridici] {Relé modul};
        \node (zasuvky) [blok, right=0.5 of rele] {Zásuvky 230VAC};
        \node (uzel) [style={draw, circle, minimum size=0.1cm, fill}, above=1cm of zasuvky] {};
        \node (sit) [blok, above= of uzel ] {Přívod 230VAC};
        \node (display) [blok, below left=-0.5cm and 0.65cm of ridici] {Status display};
        \node (ledstrip) [blok, above left=-0.5cm and 0.65cm of ridici] {Status LED pásek};

        \node (sbirnice1) [blok, align=center, below left=2.2cm and -1.2cm of ridici] {Sběrnice pro periferie \\ \#1};
        \node (sbirnice2) [blok, align=center, below right=2.2cm and -1.2cm of ridici] {Sběrnice pro periferie \\ \#2};
    
        % Spojovací linie
        \draw[-] (sit) -- (uzel);
        \draw[-] (uzel) -- (napajeni);
        \draw[-] (uzel) -- (zasuvky);
        \draw[-] (napajeni) -- (ridici);
        \draw[-] (ridici) -- (napajeni);
        \draw[-] (ridici) -- (display);
        \draw[-] (ridici) -- (ledstrip);
        \draw[-] (ridici) -- (sbirnice1);
        \draw[-] (ridici) -- (sbirnice2);
        \draw[-] (ridici) -- (rele);
        \draw[-] (rele) -- (zasuvky);
    
        % Obdélník, který obklopí vybrané uzly
        \node (hlavni-cast) [rect={(napajeni) (display) (ledstrip) (zasuvky)}] {};
        \node[label,above left=0.2cm and -3.6cm of hlavni-cast] {Hlavní šasi zařízení};
    \end{tikzpicture}
    
    \caption{Blokové schéma systému.}
    \label{fig:blokove-schema}
\end{figure}

